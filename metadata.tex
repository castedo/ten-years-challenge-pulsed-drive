% DO NOT EDIT - automatically generated from metadata.yaml

\def \codeURL{https://github.com/sabinomaggi/ten-years-challenge-pulsed-drive}
\def \codeDOI{https://doi.org/10.1063/1.362395}
\def \codeSWH{https://archive.softwareheritage.org/swh:1:cnt:94a9ed024d3859793618152ea559a178aadbb6f2}
\def \dataURL{}
\def \dataDOI{}
\def \editorNAME{}
\def \editorORCID{}
\def \reviewerINAME{}
\def \reviewerIORCID{}
\def \reviewerIINAME{}
\def \reviewerIIORCID{}
\def \dateRECEIVED{01 March 2020}
\def \dateACCEPTED{}
\def \datePUBLISHED{}
\def \articleTITLE{[Re] Reproduction of Step width enhancement in a pulse-driven Josephson junction}
\def \articleTYPE{Replication}
\def \articleDOMAIN{Physics}
\def \articleBIBLIOGRAPHY{bibliography.bib}
\def \articleYEAR{2020}
\def \reviewURL{}
\def \articleABSTRACT{When I read about the Ten Years Challenge, it was a matter of seconds to take the plunge and decide to partecipate with the replication of a work of a quarter of century ago about the numerical simulation of a Josephson junction driven by a train of microwave pulses. 
I was fairly dubious to be able to replicate the results of that work, but in the end the effort resulted much easier than I initially thought. 
The paper describes the tools and tricks used for the replication and the lessons learned from this work.
Among them, keeping good and updated documentation is paramount. Likewise, the use of tools and languages that do not fall into oblivion after one or two iterations.}
\def \replicationCITE{}
\def \replicationBIB{}
\def \replicationURL{}
\def \replicationDOI{}
\def \articleKEYWORDS{rescience c, fortran, visual basic, josephson junction, voltage standard}
\def \journalNAME{ReScience C}
\def \journalVOLUME{4}
\def \journalISSUE{1}
\def \articleNUMBER{}
\def \articleDOI{}
\def \authorsFULL{Sabino Maggi}
\def \authorsABBRV{S. Maggi}
\def \authorsSHORT{Maggi}
\title{\articleTITLE}
\date{}
\author[1,\orcid{0000-0002-1523-6484}]{Sabino Maggi}
\affil[1]{National Research Council, Institute of Atmospheric Pollution Research, CNR-IIA, Bari, Italy}
